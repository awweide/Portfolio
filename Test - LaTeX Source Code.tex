\documentclass [a4paper ,10 pt]{article}
% <<< Main packages
\usepackage{graphicx} 
\usepackage{fullpage}
\usepackage{amsmath}
\usepackage{amsthm} 
\usepackage[latin1]{inputenc}
\usepackage[norsk]{babel}
%\usepackage[lf]{venturis}
% >>>
% <<< Margins. Wider
\setlength{\topmargin}{-.0in}
\setlength{\textheight}{9in}
\setlength{\oddsidemargin}{.125in}
\setlength{\textwidth}{6.25in}
% >>>
% <<< Theorem environment. Unified numbering
\newtheorem{theorem}{Theorem}
\newtheorem{lemma}[theorem]{Lemma}
\newtheorem{proposition}[theorem]{Proposition}
\newtheorem{corollary}[theorem]{Corollary}
\newtheorem{q}{Oppgave}

\setcounter{secnumdepth}{-1}
% >>>
% <<< Begin document
%\author {Haakon C. Bakka \\
%SINTEF }
\title { Pr\o ve\\ Oppfriskningskurs 2012}
\date {}

\begin {document}
\maketitle

\section{Oppgave 1}
L\o s ligningen:
\begin{equation*}
\frac{  x-\frac{1}{2}  }{x+2} = \frac{1}{\frac{7}{2}}
\end{equation*}

\section{Oppgave 2}
Anne er tre ganger s\aa \; gammel som Bent. Bent er 11 \aa r yngre enn David, som er halvparten s\aa \; gammel som Anne. Hvor gamle er Anne, Bent og David?

\section{Oppgave 3}
De tre sidene i en trekant er 5, 12 og 13 cm. Finn vinkelen mellom de to lengste sidene.

\section{Oppgave 4}
Finn alle l\o sninger av lignignen: $\ln^2 x - \ln x^4 = -4$

\section{Oppgave 5}
Bent har kj\o pt bruktbil. Den var allerede ti \aa r gammel da han kj\o pte den til 157.000,- . Biler antas \aa \; synke 10$\%$ i verdi hvert \aa r fra de er nye. Hva var verdien av bilen da den var ny?

\section{Oppgave 6}
Finn alle l\o sninger av $p(x) = x^3 - 21x + 20 = 0$ . Merk at $p(1) = 0$ .

\section{Oppgave 7}
NB: Dette er en ulikhet, ikke en ligning!\\
Finn alle l\o sninger av ulikheten: $x+4 \geq -\frac{2}{x+1}$ 

\section{Oppgave 8}
Finn alle l\o sninger av ligningen, ikke bare de hvor $\theta \in [0,2\pi )$, oppgitt i radianer: $\sin(2 \theta) = 0$

\section{Oppgave 9}
\noindent a) Finn sentrum og radius i sirkelen beskrevet av $x^2 + 10x + y^2 = 4y + 20$\\
\noindent b) Hva slags geometrisk figur beskriver $y = -\sqrt{9-x^2}$ ?

\section{Oppgave 10}
\noindent a) Du er gitt to punkter, $A=(3, 1)$, $B=(6,-3)$. Finn et punkt $C$ slik at $\bigtriangleup ABC$ er rettvinklet og  $Areal(\bigtriangleup ABC)$ = 20 .\\
\noindent b) Hvor mange ulike punkter $C$ finnes det som oppfyller kravene? Vis omtrent hvordan de er plassert ved \aa \; tegne en figur. Du trenger ikke bestemme koordinatene.


\section{Oppgave X}
Denne er ikke en del av selve pr\o ven. Den skal ikke leveres, men er noe \aa \; fundere over om du \o nsker noe mer kreativt og anderledes mens du venter p\aa \; gjennomgangen.\\
\\
Definer de to retningene p\aa \; et vanlig sjakkbrett som nord/s\o r og \o st/vest. Legg til flere sjakkbrett, slik at du har 8 niv\aa \; opp\aa \; hverandre som alle er som vanlige $8x8$ sjakkbrett. Brikker kan bevege seg opp og ned mellom etasjene p\aa \; samme m\aa te som de ellers beveger seg.\\
\\
Hvor f\aa \; t\aa rn kan du plassere ut, slik at alle felter har et t\aa rn i seg, eller kan n\aa es av et t\aa rn i ett trekk?
\end{document}
